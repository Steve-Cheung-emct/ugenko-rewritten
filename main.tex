% !TEX program = ptex2pdf
%%		uplatex main
%%		uplatex main
%%		dvipdfmx  main.dvi

\documentclass[ribbon]{genko}

% 説明
% 必選項(他們之間都是互斥的。)
% pdfm/tgenko/ygenko/ribbon/binsen/test
%
% 可選項
% 紙張選項
% landscape	選項,將紙張橫放。
% yoko		選項,設置為橫向書寫方向。
% tate		選項,設置為縱向書寫方向。(默認)
% a4paper	或	A4	選項,選擇 A4 紙張。
% a5paper	或	A5	選項,選擇 A5 紙張。
% b4paper	或	B4	選項,選擇 B4 紙張。
% b5paper	或	B5	選項,選擇 B5 紙張。(默認)
% test		選項,選擇 B5 高度的卷子。長度為4200mm,高度為257mm。測試選項,輸出卷子。

% 頁眉頁碼行爲
% 橫向頁碼	選項,調用橫向頁碼。默認為縱向書眉+頁碼。
% 名字/ID/名前	選項,輸入底部和頁碼平齊的 ID。需要【橫向頁碼】選項。

% 版式樣式
% pdfm		選項,控制騎縫書眉。包括,奇偶公用頁碼、書名、魚尾、象鼻填充等。默認 B5 ,垂直書寫。支持A4、B5.
% binsen	選項,日式便牋。pdfm 的反面,不要騎縫書眉。默認 B5 ,垂直書寫。支持A4、B5.
% tgenko	選項,調用橫置的作文紙,A4、B5 兩種紙張,排版效果相同,字號也相同。B5是由A4縮印而來。
%				%  A5 紙為15x15規格。A4、B5 皆爲 20x20;沒有B4紙張對應的選項。
% ygenko	選項,調用橫向書寫的作文紙。(待書:B4 作文紙。已决定不支持B4紙。)
% ribbon	選項,調用日式作文紙。只有 B4 一種幅面。且為 landscape、垂直書寫。
% test		選項,大唐西域記式卷子。預設長度為 4200 mm,高度為 257 mm。需要較高的 LaTeX 水平。不建議新手使用。
% pagestyle 説明
% \pagestyle{plain}			正文。内外邊框、糸欄、水平頁碼、騎縫書眉(pdfm);内外邉框、糸欄、垂直頁碼(無pdfm)。
% \pagestyle{my}			前言。只有内外邉框,沒有欄綫。
% \pagestyle{mytext}		有内外邉框,有欄綫。無書眉,無頁碼。
% \thispagestyle{chapter}	對 pdfm 模式下騎縫書眉的標題頁修復。
							%(在有糸欄的版式下使用,此爲必須且可靠的解決辦法。)
% \thispagestyle{mychapter}	對 pdfm 模式下非騎縫書眉的標題頁修復。
% \pagestyle{myabstrac}		pdfm 選項下使用的騎縫凡例。無 pdfm 選項則爲垂直書眉的凡例。
% \pagestyle{mymenu}		pdfm 選項下使用的騎縫目錄。無 pdfm 選項則爲垂直書眉的目錄。
% \pagestyle{mymyappendix}	pdfm 選項下使用的騎縫附錄。無 pdfm 選項則爲垂直書眉的附錄。

% 嘗試畫一個表格來説明這種選項的搭配。已寫。

% 载入自定义设置
\usepackage{genkosettings}

% 載入的字體配置信息
\usepackage{genkofonts}

% 設置的直書冒號禁則。
\prebreakpenalty`︰=10000

%%%%%%%%%%%%%%%%%%%%%%%%%%%%%%%%%%%%%%%
% 載入的紙張配置信息。此係必須文件。     %%
\input{pagesizedef.clo}              %%
%%%%%%%%%%%%%%%%%%%%%%%%%%%%%%%%%%%%%%%

\maintitle{倚天屠龍記}
\subtitle{連載版}
\author{金 庸}
\authorfn{金 庸}

\usepackage{lineno}
\maxdeadcycles=2000
% 行數 x 35
%\setlength{\TEXTLENGTH}{-2469.44 mm}

%\voffset=-10 mm
%\hoffset=-100 mm
\begin{document}

\Large
 \pagestyle{empty}
\setlength{\parindent}{ 0 pt}





\pagestyle{plain}
%\linenumbers

\ugenkokaib
\normalsize
%大唐西域記 \hfill{尚書左僕射燕國公于志寧}\hspace*{2zw} 

\input{test.tex}


\end{document}








\userelfont\ujlreq

%%%%%% 封面 %%%%%%
\pagecolor{konjou}

%\maketitle

%\cleardoublepage

%\setcounter{page}{-9}
\pagecolor{white}
%\pagecolor{ikkonzome} % 一斤染
%\pagecolor{sakurairo} % 樱色 少女粉
%\pagecolor{shiou!60}

 \pagestyle{empty}



\begin{center}
\begin{minipage}<y>[htpb]{160mm}
\fontsize{12pt}{18}\selectfont\ttfamily
{\LARGE\centering\gt 簡易説明\\}

\空行

此預覽版使用之前需配置用戶本地字體,在 genkofonts.sty 中,125 行起。

將其中的字體替換爲用戶本地字體,格式如示例。

日文字體使用 UniJIS2004-UTF16-H/V 作爲 CMAP 映射。

中文字體使用 UniGB-UTF16-H/V 作爲映射,或者使橫書字體使用 unicode ,

直書字體使用 unicode 竝使用 -w 1 參數,將漢字參考方向旋轉90°。


\空行

{\LARGE\centering\gt 項目文件簡要描述\\}

\空行

\begin{biao}[項目文件簡要描述文件]
\item[genko.cls] class文件。對 菅野 善久 老師的 genko (原稿紙)的重寫。
\item[colordef.clo] class 使用的顔色定義文件。
\item[pdfm.clo] 騎縫書眉使用的罣綫、邉框、書眉、頁碼等的tikz定義。
\item[binsen.clo] 便箋紙使用的定義,pdfm 的否定狀態。
\item[tgenko.clo] 垂直書寫作文紙的網格定義。
\item[ygenko.clo] 水平書寫作文紙的網格定義。
\item[ribbon.clo] 日式JIS B4作文紙的網格定義。
\item[test.clo] 大唐西域記式卷子本的定義。每製作一次卷子,都需要重新定義。
\item[pagesizedef.clo] class 使用的外部判斷,主要是判斷紙張尺寸,以及對應的平衡參數。
\item[genkoid.tex] 頁脚使用的姓名文本。用戶自定義。
\item[genkofonts.sty] 自用字體的 NFSS2 配置。主要針對作文紙一字一格,使用刪掉glue和kern表的虛擬字體。
\item[genkosettings.sty] 配套的設置文件。
\item[tochu.sty] 頭注包文件。本模板未使用。若使用頭注包,則需重構 pagesizedef.clo 中的 voffset 參數。
\item[warichus.sty] 割注包文件。本模板未使用。
\item[jcolor.sty] 日式顔色包文件。
\item[jcolor.tex] 日式顔色包説明文件。
\item[main.tex] 本文。
\item[test.tex] 測試文本。
\item[TFM文件夾] 自製虛擬字體向量信息文件(TFM)。
\item[VF 文件夾] 自製虛擬字體。安裝方法,將此兩個文件夾拷貝到 \verb+C:\texlive\texmf-local\fonts+ %
下對應的目錄中,再執行 \verb+mktexlsr+。
\end{biao}
\end{minipage}
\end{center}

\end{document}






%\vspace*{\baselineskip}
\begin{center}
\begin{minipage}<y>[htpb]{180mm}
\fontsize{12pt}{18}\selectfont\ttfamily
{\LARGE\centering\gt 本項目支持的紙張及選項明細}\\[2mm]
\begin{tabular}{|c|c|c|c|c|c|c|c|c|}
\hline

項目 & {\color{shuiro}pdfm} & {\color{shuiro}binsen} & {\color{shuiro}tgenko} & {\color{shuiro}ygenko} & {\color{shuiro}ribbon} & {\color{shuiro}作文紙} & {\color{shuiro}test} & 注 \\ \hline
a3paper & ╳ & ╳ & ╳ & ◯ & ╳ & ◯ &  ╳ & \\ \hline
a4paper & ◯ & ◯ & ◯ & ◯ & ╳ & ╳ & ╳ & \\ \hline
a5paper & ╳ & ╳ & ◯ & ◯ & ╳ & ╳ & ╳ & \\ \hline
b4paper & ╳ & ╳ & ╳ & ◯ & ◯ & ╳ & ╳ & \\ \hline
b5paper & ◯ & ◯ & ◯ & ◯ & ╳ & ╳ & ╳ & \\ \hline
landscape & ╳ & ╳ & ◯ & ◯ & ◯ & ◯ & ◯ & 紙張橫置 \\ \hline
tate & ◯ & ◯ & ◯ & ╳ & ◯ & ╳ & ◯ & 直書 \\ \hline
yoko & ╳ & ╳ & ╳ & ◯ & ╳ & ◯ & ◯ & 橫書 \\ \hline
橫向頁碼 & ◯ & ◯ & ◯ & ◎ & ◯ & ╳ & ╳ & \\ \hline
名字 & ◯ & ◯ & ◯ & ◎ & ◯ & ╳ & ╳ & \\ \hline
test & ╳ & ╳ & ╳ & ╳ & ╳ & ╳ & ◯ & 卷子本 \\ \hline
注 & 騎縫書眉 & 日式便箋 & 直書作文紙 & 橫書作文紙 & 日式B4作文紙 & 考試用作文紙 & 卷子本& \\ \hline
古典書式 & 永樂大典式 & 紅樓夢式 & & &  & & 大唐西域記式 &  \\ \hline

\end{tabular}

\vskip5mm
注1:表中,支持用 ◯ 表示,不支持用 ╳ 表示。部分支持用 ◎ 表示。\\[1mm]
注2:紅色為必選項,且它們之間為互斥関係。\\[1mm]
注3:四庫全書式過幾天再寫。
\end{minipage}
\end{center}

\end{document}







\begin{center}
\begin{minipage}<y>[htpb]{160mm}
\fontsize{12pt}{18}\selectfont\ttfamily

{\LARGE\centering\gt 支持調用的選項\\}

\空行

\begin{biao}[項目文件]
\item[b5paper,tate,final,openleft,twoside,onecolumn,淺朱,LightRed] %
\item[] 默認選項:B5 紙張,直書,終稿,左開,雙邉,單欄;書眉文字為淺朱,欄目顔色為淺朱(LightRed)。
 此默認選項不含版式,所以必須再選擇版式,否則將出錯。
\item[pdfm] 騎縫書眉。此版式為永樂大典式,葉 16 欄,每欄兩行,行 32 字,行首空四格。
欄寬 52 pt,行寬 26 pt。頁 448 字,葉 896 字。若使用頭注,則需重構 pagesizedef.clo 中的 voffset 參數。
只支持 A4 B5 紙張,使用其他紙張將出錯。
\item[binsen] 便箋紙使用的定義,pdfm 的否定狀態。不使用騎縫書眉。
只支持 A4 B5 紙張,使用其他紙張將出錯。
\item[tgenko] 直書作文紙,默認 landscape 選項。頁面規格 20x20.
只支持 A4 B5 A5 紙張,使用其他紙張將出錯。B5 紙張由 A4 縮放得到。
A4 B5 頁面規格 20x20. A5 頁面規格 15x15.
\item[ygenko] 橫書作文紙,支持 A3 A4 A5 B5 紙張。選擇 B4 紙張將出錯。
其中 A3 默認 landscape 選項。B5 紙張由 A4 縮放得到。
A4 B5 頁面規格 20x20. A5 頁面規格 15x15. 此 A3 由兩張 A4 拼合而成。
頁面規格也是 20x20. 整版共 800 字。不支持 B4 紙張。
\item[作文紙] ygenko A3 紙張的增强版,用於對考試作文紙的模擬。頁面規格 20x23. 
行高 30 pt。整版共 920 字。
如果調用帶有章標題的頁面佈局(\verb+\pagestyle{chapter}+),則整版共 880 字。
\item[ribbon] 日式 JIS B4 作文紙。 每葉 20 欄 20 行,每行 20 字,欄寬 42 pt。
對於所有帶網格的版式,使用 \verb+\空行+ 產生一個空行,此命令後多敲兩個 Enter ,
使下一行成爲一個段落。否則將引起異常。此命令等效爲\verb+\vskip\baselineskip+;
使用 \verb+\空一格+ 、 \verb+\空兩格+ 對行首進行空格處理。
默認空格請設置爲 0 pt,否則段落首行將無法對齊。
\item[test] 大唐西域記式卷子本。它的技術難點在於尋找文本的起點,以及網格的起點。
網格寬度即爲行高 35 pt,網格從右往左繪製。第一條網格綫即爲版心的右側界限。 
每製作一次卷子,都需要重新定義 pagesizedef.clo 中對應的平衡參數。
\item[橫向頁碼] 或“横向頁碼”“橫頁碼”“横頁碼”“yokopage”:調用頁脚頁碼。默認無頁脚。
\item[名字] 或“ID”“名前”“myname”:調用頁脚簽名。需要“橫向頁碼”支持。
\item[顔色選項] 詳見 colordef.clo,其中漢字選項多爲書眉漢字顔色定義。假名選項為網格顔色。
\end{biao}
\end{minipage}
\end{center}


\end{document}